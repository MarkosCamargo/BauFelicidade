\documentclass[	english,			% idioma adicional para hifenização
	brazil,]{beamer}
%
% Choose how your presentation looks.
%
% For more themes, color themes and font themes, see:
% http://deic.uab.es/~iblanes/beamer_gallery/index_by_theme.html
%
  \usetheme{Boadilla}      % or try Darmstadt, Madrid, Warsaw, ...
  \usecolortheme{default} % or try albatross, beaver, crane, ...
  \usefonttheme{default}  % or try serif, structurebold, ...
  \setbeamertemplate{navigation symbols}{}
  \setbeamertemplate{caption}[numbered]

\usepackage[english]{babel}
\usepackage[utf8x]{inputenc}

\title[BAÚ DA FELICIDADE]{BAÚ DA FELICIDADE ABORDADO PELO PROBLEMA DA MOCHILA DE MÚTIPLA ESCOLHA}
\author[]{GABRIEL NAOTO~YMAI~PEREIRA \and JANDIR~LUIZ~HABITZREUTER \and LUMA~TURATTO~HOSCHER \and MARCOS~RUFINO~DE~CAMARGO \and SILA~GEORGES~AGIRÚ~JUDICK~SIEBERT}
\institute{UDESC-CEAVI}
\date{Ibirama, 23 de Março 2016}

\begin{document}

\begin{frame}
  \titlepage
\end{frame}

% Uncomment these lines for an automatically generated outline.
%\begin{frame}{Outline}
%  \tableofcontents
%\end{frame}

\section{O CENÁRIO}

\begin{frame}{O CENÁRIO}

Nossa empresa foi procurada por um grande empresário que deseja lançar um produto inovador, uma cesta de produtos surpresa denominada provisoriamente de “Baú da Felicidade”. Segundo o empreendedor Senhor Abravanel, os clientes pagam uma mensalidade e a cada trimestre eles recebem uma cesta com produtos surpresa. O cliente ainda não tem claro o funcionamento de todo o processo. Contudo já obtivemos algumas premissas extraídas de uma conversa informal.


\end{frame}

\subsection{PRODUTOS}

\begin{frame}{PRODUTOS}

\begin{itemize}
\item A compra dos produtos será através de um processo composto por uma tomada de preços e a escolha dos produtos. A tomada de preço é similar a um leilão fechado. Será lançado uma intenção de compra e os fornecedores submetem seu preço a cada um dos produtos. Havendo concordância com o valor apresentado pelo fornecedor a compra é efetuada.
\item Os produtos são separados por categorias.
\item A cesta só pode conter produtos que não tenham sido utilizados em outras cestas por um período de 1 ano.
\end{itemize}

% Commands to include a figure:
%\begin{figure}
%\includegraphics[width=\textwidth]{your-figure's-file-name}
%\caption{\label{fig:your-figure}Caption goes here.}
%\end{figure}

\end{frame}

\subsection{CESTA}

\begin{frame}{CESTA}

\begin{itemize}
	
\item A princípio, uma cesta não pode conter mais que um produto de uma categoria.
%\pause
\item A cesta deve ter o menor custo possível.
%\pause
\item Deve ser levado em consideração a satisfação que cada produto proporciona aos clientes.
%\pause
\item Desconsiderar a quantidade de produtos da cesta, apenas certificar-se de que o valor total da cesta não ultrapasse o valor máximo estipulado para a cesta do trimestre.
\item Para resolução do problema deve-se arredondar os preços dos produtos para cima, caso os mesmos possuam casas decimais, para que os preços sejam tratados como números inteiros e garantir que o domínio matemático do problema seja discreto.
\item O problema deve ser tratado como um Problema da Mochila de Múltipla Escolha.
\end{itemize}

\end{frame}

\subsection{ENTREGA}

\begin{frame}{ENTREGA}
\begin{itemize}
\item O entregador tem que fazer viagens otimizadas e gastar o mínimo de tempo possível. Ao final o mesmo deve regressar ao depósito da empresa.
\item A entrega será sempre realizada no terceiro dia após o pagamento da terceira mensalidade.
\item Cada entregador pode trabalhar no máximo por 6 horas nas entregas.
\end{itemize}
\end{frame}

\section{PROBLEMA DA MOCHILA}
\begin{frame}{PROBLEMA DA MOCHILA}
\begin{itemize}
\item O problema da mochila é um dos 21 problemas NP-completos de Richard Karp, exposto em 1972. A formulação do problema é extremamente simples, porém sua solução é mais complexa.

\item	O cenário é onde se tem um conjunto de itens a serem colocados em uma mochila. Nesse problema, existe uma determinada quantidade de itens, cada qual com o seu peso e valor, onde deseja-se colocar esses itens em uma mochila com uma capacidade predefinida. O objetivo é colocar os itens na mochila de modo a se obter o maior valor (composto pela soma de valores de cada item inserido na mochila), desde que não ultrapasse o peso total suportado pela mochila.
\end{itemize}
\end{frame}


\section{PROBLEMA DA MOCHILA APLICADO AO BAÚ DA FELICIDADE}
\begin{frame}{PROBLEMA DA MOCHILA APLICADO AO BAÚ DA FELICIDADE}
Comparando o problema do “Baú da Felicidade” e o da Mochila, pode-se afirmar que ambos possuem três variáveis. No “Baú da Felicidade” temos o valor máximo da Cesta (equivalente ao peso máximo da Mochila) e os objetos possuem valor(equivale ao peso do objeto da Mochila) e satisfação(equivalem ao preço dos objetos da Mochila).
Os métodos abordados nesta pesquisa são: solução usando Programação Dinâmica, solução usando o Método Guloso e solução usando relaxação Lagrangeana.
\end{frame}

\subsection{PROBLEMA DA MOCHILA DE MÚLTIPLA ESCOLHA}
\begin{frame}{PROBLEMA DA MOCHILA DE MÚLTIPLA ESCOLHA}
	O Problema da Mochila Compartimentada que é uma variação do clássico problema da mochila e pode ser enunciado considerando-se a seguinte situação hipotética: um alpinista deve carregar sua mochila com possíveis itens de seu interesse. A cada item atribui-se o seu peso e um valor de utilidade (até aqui, o problema coincide com o clássico Problema da Mochila ). Entretanto, os itens são de agrupamentos distintos (alimentos, medicamentos, utensílios, etc.) e devem estar em compartimentos separados na mochila. O problema consiste em determinar as capacidades adequadas de cada compartimento e como esses devem ser carregados, maximizando o valor de utilidade total.
\end{frame}


\subsection{PROBLEMAS NP-COMPLETOS}
\begin{frame}{PROBLEMAS NP-COMPLETOS}
	Na teoria da complexidade computacional, a classe de complexidade é o subconjunto dos problemas NP de tal modo que todo problema em NP se pode reduzir, com uma redução de tempo polinomial, a um dos problemas NP-completo. Pode-se dizer que os problemas de NP-completo são os problemas mais difíceis de NP e muito provavelmente não formem parte da classe de complexidade P. 
	A razão é que se conseguisse encontrar uma maneira de resolver qualquer problema NP-completo rapidamente (em tempo polinomial), então poderiam ser utilizados algoritmos para resolver todos problemas NP rapidamente. Na prática, o conhecimento de NP-completo pode evitar que se desperdice tempo tentando encontrar um algoritmo de tempo polinomial para resolver um problema quando esse algoritmo não existe.
	
\end{frame}

\section{POSSÍVEIS SOLUÇÕES}

\subsection{MÉTODO GULOSO}
\begin{frame}{MÉTODO GULOSO}
	Técnica utilizada para problemas de otimização. Sempre faz a escolha que parece melhor no momento. Sugere construir uma solução através de uma sequência de passos, cada um expandindo uma solução parcialmente construída até o momento, até ser obtida uma solução completa para o problema.
	Em cada passo, a escolha deve ser feita: 
	\begin{itemize}
		\item Possível - Deve satisfazer as restrições do problema.
		\item Localmente ótima – Deve ser a melhor escolha local dentre todas as disponíveis.
		\item Irreversível – Uma vez feita, ela não pode ser alterada nos passos seguintes do algoritmo.
	\end{itemize}
	Há expectativas de que escolhas locais ótimas levem a uma solução ótima global para o problema como um todo. Por ser um algoritmo que usa estratégia gananciosa, faz sempre escolhas que, naquele instante, parecem excelentes. Isto pode levar a uma solução ótima, ou não, mas provavelmente não vai levar a uma solução insatisfatória.
\end{frame}


\subsection{PROGRAMAÇÃO DINÂMICA}
\begin{frame}{PROGRAMAÇÃO DINÂMICA}
	Programação dinâmica é um método para a construção de algoritmos para a resolução de problemas computacionais, em especial os de otimização combinatória. Ela é aplicável a problemas nos quais a solução ótima pode ser computada a partir da solução ótima previamente calculada e memorizada - de forma a evitar recálculo - de outros subproblemas que, sobrepostos, compõem o problema original.
	O que um problema de otimização deve ter para que a programação dinâmica seja aplicável são duas principais características: subestrutura ótima e superposição de subproblemas. Um problema apresenta uma subestrutura ótima quando uma solução ótima para o problema contém em seu interior soluções ótimas para subproblemas. A superposição de subproblemas acontece quando um algoritmo recursivo reexamina o mesmo problema muitas vezes.
\end{frame}


\subsection{RELAXAÇÃO LAGRANGEANA}
\begin{frame}{RELAXAÇÃO LAGRANGEANA}
	A relaxação lagrangeana é uma técnica para se obter limitantes duais de problemas combinatórios que podem ser modelados como programas lineares inteiros.
\end{frame}

\section{CONCLUSÃO}
\begin{frame}{CONSIDERAÇÕES FINAIS}
A princípio, para testes iniciais, utilizaremos o método guloso, mas como apresentado este método não é tão eficiente comparado a outros, como por exemplo, o método da programação dinâmica, que será nossa segunda tentativa de abordagem do problema.
\end{frame}
\end{document}
